We define a Financial Factor as the excess return of an implementable strategy (random variable!).

\note Long in value firm, short in growth firm: Let's assume that you go long in asset A and go short in asset B. Hence, 
$$
		r^A - r^B  = (r^A - r^f) - (r_B-r^f).	
$$
All long-short positions can be interpreted as generated by financial factors. All we are trying to say is that the financial factors must be represented like
$$
	F_t^i = \bar{F}_t + \beta^i \cdot F_t,
$$
where $\beta^i, F_k\in\R^K$. Whatever we have in the left hand side (one financial factor of a particular firm, i.e., $F_t^i\in\R$) it can be expressed in terms of $F_t$. Where in this case $\beta^i = \text{e}_i$. Hence, since the the beta model implies that
$$
	\E_t[F_{t+1}^i] = \beta^i  \lambda .
$$
Hence, 
$$
		\lambda^i = \E[F_{t+1}^i].
$$
Why do we like financial factors? Because they are easy to estimate: we make a regression with just a constant.

And what about the $b$'s? Note that $\hat{b} = \hat{\lambda}\Sigma^{-1}$ where
$$
	\V(\hat{b}) = \Sigma^{-1}\V(\lambda)\Sigma^{-1}.
$$
Hence
$$
SE(\hat{b}) = \sqrt{\text{diag}\left(\V(\hat{b})\right)}
$$
We love financial factors, $\lambda$, because we know a lot about them.