Do we learn something from portfolio formation? Do we lose info from portfolio? We are gonna do the following:
$$
	\begin{aligned}
	 P_t^i = \E_t[M_{t+1}(P_{t+1}^i + D_{t+1}^i)]\iff X_t P_t^i = \E_t[X_t M_{t+1}(P_{t+1}^i + D_{t+1}^i)],
	\end{aligned}
$$
where $X_t$ is the time-varying conditioning information randim variable.

Let's consider a managed portfolio, i.e., $((X_t^i)_i)_t$, for some assets $i=1,\ldots, N$. An example could be the following:

Let's consider two firms and let's look at their book-to-market, which is something that we can compute (equity value according to accounting criteria...)
\begin{table}[h!]
	\centering
	\begin{tabular}{c||ccc}\hline
		B/M & 1 & 2 & 3\\\hline\hline
		1 & 1 & 1 & 8\\\hline
		2 & 5 & 7 & 1\\\hline
	\end{tabular}
\end{table}
Let's consider a strategy consisting in investing in ``value'' (not ``growth'') firms. Hence, 
\begin{table}[h!]
	\centering
	\begin{tabular}{c||ccc}\hline
		 &1 & 2 & 3\\\hline\hline
		$\chi^{i=1}$ & 1 & 1 & 0\\\hline
		$\chi^{i=2}$ &0 & 0 & 1\\\hline
	\end{tabular}
\end{table}
For that reason, the value of the portfolio is given by
$$
	P_t^{\text{portfolio}} = \sum_{i=1}^N\chi_t^i p_t^i = \sum_{i=1}^N \E_t[M_{t+1}(P_{t+1}^i + D_{t+1}^i)\chi_t^i],
$$
i.e., absolutely consistent with NA.